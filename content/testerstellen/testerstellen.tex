%%%%%%%%%%%%%%%%%%%%%%%%%%%%%%%%%%%%%%%%%%%%%%%%%%%%%%%%%%%%%%%%%%%%%%%%%%%%%%%
% Titel:   Schlusswort
% Autor:   
% Datum:   
% Version: 0.0.0
%%%%%%%%%%%%%%%%%%%%%%%%%%%%%%%%%%%%%%%%%%%%%%%%%%%%%%%%%%%%%%%%%%%%%%%%%%%%%%%
%
%:::Change-Log:::
% Versionierung erfolgt auf folgende Gegebenheiten: -1. Release Versionen
%                                                   -2. Neue Kapitel
%                                                   -3. Fehlerkorrekturen
%
% 0.0.0       Erstellung der Datei
%%%%%%%%%%%%%%%%%%%%%%%%%%%%%%%%%%%%%%%%%%%%%%%%%%%%%%%%%%%%%%%%%%%%%%%%%%%%%%%
\chapter{Erstellen eines JUnit-Tests}\label{ch:testerstellen}
	Seit der Version 3.2 ist \textsf{JUnit} ein fester Bestandteil der \textsf{Eclipse IDE}. Somit ist das Testen von Software in \textsf{Eclipse} einfacher denn je. Eine Klasse kann bei der Erstellung der Tests selektiert werden und der Benutzer kann ohne jeglichen Code schreiben zu m�ssen die Grobstruktur des Test generieren. In der Folgenden Anleitung werden die einzelnen Schritte der Reihe nach erl"autert.\par
	%
	Die folgende Anleitung wird anhand eines simplen Rechners als Beispiel veranschaulicht. Mit Hilfe dieses Modells werden die grundlegenden Schritte f�r das erfolgreiche Erstellen eines Test beschrieben.
	%
	\section{Calculator Modell}\label{s:calculatormodell}
		Wie Bereits erw�hnt, wird als einfaches Beispiel ein Rechner umgesetzt. Um die Funktionsweise eines Unit-Test zu verdeutlichen werden zwei Methoden implementiert:
		%
		\begin{description}
			\item[add()] Die Methode \textsf{add()} addiert zwei �bergebene Integer und gib das Resultat zur"uck.
			\item[multyply()] Diese Methode multipliziert zwei Variablen und liefert das Ergebnis zur"uck. 
		\end{description}
		%
		Die Klasse mit diesen beiden simplen Methoden wird, wie in Listing \ref{list:calculatorclass} abgebildet, in einem neuen Java-Projekt implementiert.

\begin{lstlisting}[style=JAVA,caption={Calculator Klasse},label={list:calculatorclass}]
package ApplicationCode;

public class Calculator {
	
	public int add(int num1, int num2){
		return num1 + num2;
	}
	
	public int multiply(int num1, int num2){
		return num1 * num2;
	}

}
\end{lstlisting}
		%
	\section{Schritt f"ur Schritt Anleitung}\label{s:anleitung}
		Auf den Folgenden Seiten wird Schritt f"ur Schritt auf die einzelnen Arbeitsschritte eines erfolgreichen Unit-Test mit \textsf{JUnit} eingegangen. Dabei wird das das in \ref{s:calculatormodell} auf Seite \pageref{s:calculatormodell} beschriebene Modell getestet.
		%
		\subsection*{Ordnerstruktur}\label{s:ordnerstruktur}
			Um ein Unit-Test zu erstellen empfiehlt es sich innerhalb eines Projektes eine klare Struktur f�r die Tests aufzubauen. Eine gute M"oglichkeit die Test vom "ubrigen Code Abzutrennen ist ein zus�tzlicher Source-Folder. Diese kann via \textit{File $\rightarrow$ New $\rightarrow$ Source Folder} erstellt werden. Der neue Ordner sollte einen etwas aussagekr�ftigen Namen wie z.B. Test, UnitTest oder etwas in diese Richtung haben, damit eine andere Person einfach erkennen kann, f�r was dieser Ordner erstellt wurde.
			%
		\subsection*{Erstellen einer Testklasse}\label{s:testklasseerstellen}
		Nach dem Erstellen des neuen Ordners kann eine neue Testklasse erstellt werden. Dazu wird mit \textit{File $\rightarrow$ New $\rightarrow$ Other...} das Kontextmen"u (Abbildung \ref{s:calculatormodell}) f�r neue Klassen aufgerufen werden.\par
		%
		\image{content/testerstellen/image/AuswahlJUnit.png}{scale=.4}{htbp}[Literaturverzeichnis erstellen]
		%
		Im Suchfeld kann Nach \textsf{JUnit} gesucht werden. Durch selektieren von JUnit Test Case und Klicken auf Next, wird das n�chste Fenster ge�ffnet.
		%
	\section{Zusammenfassung}\label{s:zusammenfassung}
		Die einzelnen Schritte werden hier noch einmal kurz zusammengefasst:
		%
		\begin{enumerate}
			\item Source-Folder f�r Tests erstellen unter \textit{File $\rightarrow$ New $\rightarrow$ Source Folder}.
		\end{enumerate}