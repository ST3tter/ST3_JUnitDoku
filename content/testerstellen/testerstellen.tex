%%%%%%%%%%%%%%%%%%%%%%%%%%%%%%%%%%%%%%%%%%%%%%%%%%%%%%%%%%%%%%%%%%%%%%%%%%%%%%%
% Titel:   Schlusswort
% Autor:   
% Datum:   
% Version: 0.0.0
%%%%%%%%%%%%%%%%%%%%%%%%%%%%%%%%%%%%%%%%%%%%%%%%%%%%%%%%%%%%%%%%%%%%%%%%%%%%%%%
%
%:::Change-Log:::
% Versionierung erfolgt auf folgende Gegebenheiten: -1. Release Versionen
%                                                   -2. Neue Kapitel
%                                                   -3. Fehlerkorrekturen
%
% 0.0.0       Erstellung der Datei
%%%%%%%%%%%%%%%%%%%%%%%%%%%%%%%%%%%%%%%%%%%%%%%%%%%%%%%%%%%%%%%%%%%%%%%%%%%%%%%
\chapter{Erstellen eines JUnit-Tests}\label{ch:testerstellen}
	Seit der Version 3.2 ist \textsf{JUnit} ein fester Bestandteil der \textsf{Eclipse IDE}. Somit ist das Testen von Software in \textsf{Eclipse} einfacher denn je. Eine Klasse kann bei der Erstellung der Tests selektiert werden und der Benutzer kann ohne jeglichen Code schreiben zu m�ssen die Grobstruktur des Test generieren lassen. In der Folgenden Anleitung werden die einzelnen Schritte der Reihe nach erl"autert.\par
	%
	\section{Calculator Modell}\label{s:calculatormodell}
		Die Anleitung wird anhand eines simplen Rechners als Beispiel veranschaulicht. Mit Hilfe dieses Modells werden die grundlegenden Schritte f�r das erfolgreiche Erstellen eines Test beschrieben. Um die Funktionsweise eines Unit-Test zu verdeutlichen werden zwei Methoden implementiert:
		%
		\begin{description}
			\item[add()] Die Methode \textsf{add()} addiert zwei �bergebene Integer und gib das Resultat zur"uck.
			\item[multyply()] Diese Methode multipliziert zwei Variablen und liefert das Ergebnis zur"uck. 
		\end{description}
		%
		Die Klasse mit diesen beiden simplen Methoden wird, wie in Listing \ref{list:calculatorclass} abgebildet, in einem neuen Java-Projekt implementiert.

\begin{lstlisting}[style=JAVA,caption={Calculator Klasse},label={list:calculatorclass}]
package ApplicationCode;

public class Calculator {
	
	public int add(int num1, int num2){
		return num1 + num2;
	}
	
	public int multiply(int num1, int num2){
		return num1 * num2;
	}

}
\end{lstlisting}
		%
	\section{Schritt f"ur Schritt Anleitung}\label{s:anleitung}
		Auf den Folgenden Seiten wird Schritt f"ur Schritt auf die einzelnen Arbeitsschritte eines erfolgreichen Unit-Test mit \textsf{JUnit} eingegangen. Dabei wird das das in \ref{s:calculatormodell} auf Seite \pageref{s:calculatormodell} beschriebene Modell getestet.
		%
		\subsection*{Ordnerstruktur}\label{s:ordnerstruktur}
			Um ein Unit-Test zu erstellen empfiehlt es sich innerhalb eines Projektes eine klare Struktur f�r die Tests aufzubauen. Eine gute M"oglichkeit die Test vom "ubrigen Code abzutrennen ist ein zus�tzlicher Source-Folder. Diese kann via \textcolor{bfhorange}{\textsf{File $\rightarrow$ New $\rightarrow$ Source Folder}} erstellt werden. Der neue Ordner sollte einen etwas aussagekr�ftigen Namen wie z.B. Test, UnitTest oder etwas in diese Richtung haben.
			%
		\subsection*{Erstellen einer Testklasse}\label{s:testklasseerstellen}
		Nach dem Erstellen des neuen Ordners kann eine neue Testklasse erstellt werden. Dazu wird mit \textcolor{bfhorange}{\textsf{File $\rightarrow$ New $\rightarrow$ Other...}} das Kontextmen"u (Abbildung \ref{abb:auswahljunit}) f�r neue Klassen aufgerufen werden.\par
		%
		\image{content/testerstellen/image/AuswahlJUnit.png}{scale=0.4}{htbp}[Auswahl JUnit Test Case][Auswahl JUnit Test Case][abb:auswahljunit]
		%
		Im Suchfeld kann nach \textsf{JUnit} gesucht werden. Durch selektieren von JUnit Test Case und klicken auf Next, wird das n�chste Fenster ge�ffnet. Im neuen Men�, abgebildet in der Grafik \ref{abb:testcaseerstellen}, m�ssen folgende Einstellungen vorgenommen werden:
		%
		\image{content/testerstellen/image/TestCaseErstellen.png}{scale=0.4}{htbp}[Test Case Einstellungen][Einstellungen f�r den JUnit Test Case][abb:testcaseerstellen] 
		%
		\begin{enumerate}
		\item New JUnit 4 test.
		\item Auswahl des erstellten Test Source-Folders.
		\item Erstellen oder ausw�hlen eines Packages innerhalb des Ordners
		\item Name f"ur den Test eingeben (hier CalculatorTest)
		\item Wenn gew�nscht k�nnen Hilfsmethoden ausgew�hlt werden.
		\item Auswahl der Klasse, die getestet werden soll.
		\item Weiter mit dem Next-Button
		\end{enumerate}
		%
		\newpage
		Im n"achsten Fenster (Abbildung \ref{abb:auswahlmethoden}) kann ausgew�hlt werden, welche Methoden getestet werden sollen. F�r den Rechner werden die beiden Funktionen add() und multyply() getestet. Durch Klicken auf Finish wird die Testklasse erstellt.\par
		%
		\image{content/testerstellen/image/AuswahlMethoden.png}{scale=0.35}{htbp}[Auswahl der Methoden][Auswahl der zu testenden Methoden][abb:auswahlmethoden] 
		%
		Beim Erstellen der Ersten Testklasse erscheint, wie in Abbildung \ref{abb:warningjunit4} dahrgestellt, eine Meldung, da die JUnit Library noch nicht im Pfad des Projektes eingebunden ist. Die Standardeinstellung in diesem Fenster (\textsf{Perform the following action:} und der Auswahl  \textsf{Add JUnit library to the build path}) ist f�r die meisten Anwendung die richtige Wahl. Durch klicken auf OK wird die Library in den Pfad des Projektes eingebunden und JUnit kann genutzt werden. 
		%
		\image{content/testerstellen/image/WarningJunit4.png}{scale=0.35}{htbp}[Warnung JUnit nicht im Pfad][Warnung JUnit nicht im Pfad][abb:warningjunit4] 
		%
		\newpage
		Anschliessend wird die Testklasse automatisch generiert. Dabei wird f�r den Test des Rechners folgender Code erzeugt:\\
		%
\begin{lstlisting}[style=JAVA,caption={Testklasse},label={list:testklasse}]
package ApplicationCodeTest;

import static org.junit.Assert.*;
import org.junit.Test;

import ApplicationCode.Calculator;

public class CalculatorTest {

	@Test
	public void testAdd() {
		fail("Not yet implemented");
	}

	@Test
	public void testMultiply() {
		fail("Not yet implemented");
	}
}
\end{lstlisting}
		%
		Wie in Listing \ref{list:testklasse} zu sehen ist, werden f�r beide Methoden des Rechners jeweils eine Testmethode mit der \textsf{@Test} Annotation\footnote{Erkl�hrung zu den Annotationen im Kapitel \ref{ss:annotationen} auf Seite \pageref{ss:annotationen}} erstellt. Die neu Erstellten Testmethoden werden grunds�tzlich nach dem Schema \textsf{test} + \textsf{Methodenname der zu testenden Methode()} benannt. Standardm�ssig werden alle Testmethoden mit dem Aufruf \textsf{fail("Not yet implemented")} erzeugt.
		%
		\subsection*{Den Test laufen lassen}\label{ss:testlaufenlassen}
			Nach dem Erstellen der Testklasse und deren Methoden kann der Test einmal durchlaufen werden. Gestartet wird ein Test unter \textcolor{bfhorange}{\textsf{Run $\rightarrow$ Run As $\rightarrow$ JUnit Test}}.\par
			%
			Nach dem Durchlaufen des Tests erscheint auf der linken Seite in der \textsf{Eclipse IDE} ein neuer \textsf{JUnit} Tab mit den Testresultaten. Die Abbildung \ref{abb:erstertestlauf} auf der folgenden Seite zeigt diesen Tab nach dem der generierte Test ausgef"uhrt wurde. Im oberen Bereich ist zu sehen, dass zwei von zwei Test durchgef�hrt wurden (Runs: 2/2). Beide Test sind jedoch gescheitert (Failures: 2), da bei der Standardimplementation der Testmethoden der bereits erw�hnte \textsf{fail("Not yet implemented")} Aufruf generiert wurde.\par
			%
			\image{content/testerstellen/image/Fehler.png}{scale=0.5}{htbp}[Erster Testlauf][Erster JUnit Testlauf][abb:erstertestlauf] 
			%
		\newpage
		\subsection*{Tests schreiben}\label{ss:testschreiben}
			Damit die Test nicht nur eine Fehlermeldung ausgeben, muss Testcode f�r die beiden Methoden geschrieben werden. Im Listing \ref{list:testklasseinhalt} ist ein Beispiel f�r die Implementierung der Testmethoden abgebildet.
			%
\begin{lstlisting}[style=JAVA,caption={Testklasse},label={list:testklasseinhalt}]
public class CalculatorTest {

	@Test
	public void testAdd() {
		Calculator myCalculator = new Calculator();
		/* Pr�ft, die Addition von 2 + 3 gleich 5 ist */
		assertTrue("Addition test", myCalculator.add(2, 3) == 5);
	}

	@Test
	public void testMultiply() {
		Calculator myCalculator = new Calculator();
		/* Pr�ft, die Multiplikation von 3 * 5 gleich 15 ist */
		assertTrue("Multiply test", myCalculator.multiply(3, 5) == 15);
	}

}
\end{lstlisting}
		%
		%
	\newpage
	\section{Zusammenfassung}\label{s:zusammenfassung}
		Die einzelnen Schritte werden hier noch einmal kurz zusammengefasst:
		%
		\begin{enumerate}
			\item Source-Folder f�r Tests erstellen unter \textcolor{bfhorange}{\textsf{File $\rightarrow$ New $\rightarrow$ Source Folder}}.
			%
			\item Menu f�r eine neue Testklasse �ffnen mit \textcolor{bfhorange}{\textsf{File $\rightarrow$ New $\rightarrow$ Other... $\rightarrow$ JUnit Test Case}}.
			%
			\item Einstellungen (siehe Abbildung \ref{abb:testcaseerstellen} auf Seite \pageref{abb:testcaseerstellen}) f�r die Testklasse vornehmen und auf Next klicken.
			%
			\item Auswahl der zu testenden Methoden, beenden mit Klick auf Finish.
			%
			\item Die JUnit Library in das Projekt einbinden, erscheint ein Fenster mit einer Warnung.
			%
			\item Funktionalit�t der Testmethoden implementieren
			%
			\item Test Starten mit \textcolor{bfhorange}{\textsf{Run $\rightarrow$ Run As $\rightarrow$ JUnit Test}}
		\end{enumerate}