%%%%%%%%%%%%%%%%%%%%%%%%%%%%%%%%%%%%%%%%%%%%%%%%%%%%%%%%%%%%%%%%%%%%%%%%%%%%%%%
% Titel:   Schlusswort
% Autor:   
% Datum:   
% Version: 0.0.0
%%%%%%%%%%%%%%%%%%%%%%%%%%%%%%%%%%%%%%%%%%%%%%%%%%%%%%%%%%%%%%%%%%%%%%%%%%%%%%%
%
%:::Change-Log:::
% Versionierung erfolgt auf folgende Gegebenheiten: -1. Release Versionen
%                                                   -2. Neue Kapitel
%                                                   -3. Fehlerkorrekturen
%
% 0.0.0       Erstellung der Datei
%%%%%%%%%%%%%%%%%%%%%%%%%%%%%%%%%%%%%%%%%%%%%%%%%%%%%%%%%%%%%%%%%%%%%%%%%%%%%%%
\chapter{EclEmma}\label{ch:eclemma}
	\textsf{EclEmma}\cite{lit:eclemma} bietet erg�nzende Funktionen f�r das \textsf{JUnit} Test-Framework. Mit Hilfe dieser Erweiterung ist es m�glich, die Testabdeckung eines Unit-Tests zu visualisieren. Wurd ein Test mit \textsf{EclEmma} durchgef�grt, so wird jede Codezeile der Software gr�n hinterlegt, wird diese im Test durchlaufen. "Ubriger Code\footnote{Code der nicht im Test durchlaufen wurde} wird rot hinterlegt. Nach dem Test kann eine statistische Auswertung der Testabdeckung gemacht werden. Pro Klasse wird angegeben, zu wie vielen Prozent der Code durch den letzten Testdurchlauf abgedeckt war.\par
	%
	Mit den Informationen aus der statistischen Auswertung kann nun bei bedarf die der Test-Code angepasst werden, um auch die zuvor nicht abgedeckten Codebereiche zu testen. Somit kann schlussendlich die Qualit�t der Test und zugleich die des gesamten Code verbessert werden. 
	%
	\section{Installation}\label{s:installation}
		\textsf{EclEmma} ist als Plugin von der Internetseite \hyperlink{http://www.eclemma.org/installation.html}{www.eclemma.org/installation.html} aus installiert werden.
		Die Installation von \textsf{EclEmma} als \textsf{Eclipse} Plugin ist schnell und einfach gemacht.\par
		%
		\begin{enumerate}
			\item Unter \textcolor{bfhorange}{\textsf{Help $\rightarrow$ Install New Software...}} kann das Installationsmen� von \textsf{Eclipse} aufgerufen werden
			%
			\item Im Eingabefeld Work with: kann der Pfad \textcolor{bfhorange}{\textsf{http://update.eclemma.org/}} angegeben werden.
			%
			\item EclEmma in der liste ausw�hlen und auf Next klicken. 
			%
			\item Danach kann der Installationskontext gefolgt werden.
			%
		\end{enumerate}
		%
		Nach einem Neustart von \textsf{Eclipse} ist das Plugin verf�gbar und kann zusammen mit \textsf{JUnit} verwendet werden.\par
		%
	\section{Unit-Testing mit EclEmma}\label{s:unittestingmiteclemma}
		Ein implementierter Unit-Test kann nun mit der \textsf{EclEmma} Erweiterung gestartet werden. Unter \textcolor{bfhorange}{\textsf{Run $\rightarrow$ Coverage As $\rightarrow$ JUnit Test}} wird der Test mit dem Plugin zusammen ausgef�hrt.\par
		%
		In Abbildung \ref{abb:eclemmaausgabe} ist das Ergebnis, des in der Schritt f�r Schritt Anleitung erstelle Unit-Tests, mit dem \textsf{EclEmma} Plugin dahrgestellt. Dieser Tab �ffnet sich nach dem durchlaufen des Test automatisch im unteren Bereich der Entwicklungsumgebung.\par
		%
		\image{content/eclemma/image/coverage.png}{scale=0.60}{htbp}[EclEmma Ausgabe][Ausgabe der statistischen Analyse von EclEmma][abb:eclemmaausgabe]
		%
		Die  beiden Methoden \textsf{add()} und \textsf{multiply()} wurden zu 100\% abgedeckt. Nat�rlich ist ein Test der Codeabdeckung f�r solch einfache Methoden kaum erforderlich. Bei komplizierteren Methoden und Klassen kann dies jedoch sehr n�tzlich sein, um qualitativ hochwertige Test zu schreiben.
		%