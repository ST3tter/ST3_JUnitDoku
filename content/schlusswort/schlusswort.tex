%%%%%%%%%%%%%%%%%%%%%%%%%%%%%%%%%%%%%%%%%%%%%%%%%%%%%%%%%%%%%%%%%%%%%%%%%%%%%%%
% Titel:   Schlusswort
% Autor:   
% Datum:   
% Version: 0.0.0
%%%%%%%%%%%%%%%%%%%%%%%%%%%%%%%%%%%%%%%%%%%%%%%%%%%%%%%%%%%%%%%%%%%%%%%%%%%%%%%
%
%:::Change-Log:::
% Versionierung erfolgt auf folgende Gegebenheiten: -1. Release Versionen
%                                                   -2. Neue Kapitel
%                                                   -3. Fehlerkorrekturen
%
% 0.0.0       Erstellung der Datei
%%%%%%%%%%%%%%%%%%%%%%%%%%%%%%%%%%%%%%%%%%%%%%%%%%%%%%%%%%%%%%%%%%%%%%%%%%%%%%%    
\chapter{Schlusswort}\label{ch:schlusswort}
	Unit-Testing mit \textsf{JUmit} ist in der Entwicklungsumgebung \textsf{Eclipse} einfach und praktisch zu realisieren. Die Tests sind mit Hilfe der IDE schnell generiert und m�ssen lediglich noch mit dem genauen Testablauf erg�nzt werden. Das Erstellen von Unit-Tests ist mit einem Mehraufwand verbunden, welcher sich jedoch schnell einmal lohnen kann. Ein gut durchdachtes Testkonzept, kann viele Stunden an Fehlersuche einsparen.\par
	%
	Nat�rlich eignen sich Unit-Test nicht f�r jede Art von Software gleichermassen. Gerade bei Anwendungen mit Benutzeroberfl�che ist es schwer, die graphischen Klassen zu testen, da die Eingaben des Benutzers nur beschr�nkt automatisiert werden k�nnen.\par
	%
	Dieses Dokument gibt lediglich einen kleinen Einblick in die M�glichkeiten von JUnit und der Thematik des Unit-Testing. Tests k�nnen noch viel intelligenter und automatischer implementiert werden. \textsf{EclEmma} ist auch nur eine von vielen Erweiterungen, die zur Verbesserung der Qualit�t genutzt werden k�nnen.\par  
	%
	Pers�nlich kann ich \textsf{JUnit} weiter empfehlen. Die Test sind einfach realisiert und das best�rkte Vertrauen in die Software motiviert. Uu Beginn der Testphase ist der Mehraufwand nicht unbedingt aufbauend. Jedoch ist der Vorteil,welcher daraus resultiert den Aufwand wert. 