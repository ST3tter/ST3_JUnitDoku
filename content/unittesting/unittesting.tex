%%%%%%%%%%%%%%%%%%%%%%%%%%%%%%%%%%%%%%%%%%%%%%%%%%%%%%%%%%%%%%%%%%%%%%%%%%%%%%%
% Titel:   Unit-Testing
% Autor:   
% Datum:   
% Version: 0.0.0
%%%%%%%%%%%%%%%%%%%%%%%%%%%%%%%%%%%%%%%%%%%%%%%%%%%%%%%%%%%%%%%%%%%%%%%%%%%%%%%
%
%:::Change-Log:::
% Versionierung erfolgt auf folgende Gegebenheiten: -1. Release Versionen
%                                                   -2. Neue Kapitel
%                                                   -3. Fehlerkorrekturen
%
% 0.0.0       Erstellung der Datei
%%%%%%%%%%%%%%%%%%%%%%%%%%%%%%%%%%%%%%%%%%%%%%%%%%%%%%%%%%%%%%%%%%%%%%%%%%%%%%%
\chapter{Unit-Testing}\label{ch:unittesting}
	Ein Unit-Test ist ein St"uck Code, der einen bestimmten Teil einer Software testet. Ist ein Test-Code einmal geschrieben, kann dieser beliebig oft wiederholt werden. Somit kann bei "Anderungen in der zu testenden Software, der entsprechende Unit-Test durchgef"uhrt werden, um sicher zu stellen, dass keine neuen Fehler produziert wurden. 
	%
	\section{Warum Unit-Tests}\label{s:warumunittests}
		Heutzutage ist es in der Industrie praktisch unabdingbar, Unit-Test f�r die entwickelte Software zu erstellen. Meist werden diese Tests bereits in der Entwicklungsphase parallel oder sogar vor der eigentlichen Software definiert und umgesetzt. Besonders wichtig sind Unit-Tests in Software an der ganze Teams mitarbeiten. Hat ein Softwareentwickler Code geschrieben, so kann dieser "uber Nacht getestet werden. Bei erfolgreichem Test kann der Code am n�chsten Morgen f�r die "ubrigen Teammitglieder zur Verf"ugung gestellt werden.\par
		%
		Unit-Tests erh�hen somit die Qualit"at des Codes und decken neu implementierte Fehler fr�hzeitig auf. Diese k�nnen dann effizient behoben werden. Dadurch wird auch das Vertrauen in die Software erh�ht.   
		%
	\section{Einsatzgebiete}\label{s:einsatzgebiete}
		Der Begriff Unit-Test bezieht sich nicht zwingend nur auf Software-Tests.Unit-Tests k�nnen auch zum Beispiel Hardwarekomponenten testen. Dieses Dokument bezieht sich jedoch ausschliesslich auf Unit-Tests f�r Software, insbesondere f�r Java-Code.\par
		%
		Mehr Informationen "uber Tests in diversen Softwaresprachen sind in Kapitel \ref{s:xunit} auf Seite \pageref{s:xunit} zu finden.