%%%%%%%%%%%%%%%%%%%%%%%%%%%%%%%%%%%%%%%%%%%%%%%%%%%%%%%%%%%%%%%%%%%%%%%%%%%%%%%
% Titel:   Unit-Testing
% Autor:   
% Datum:   
% Version: 0.0.0
%%%%%%%%%%%%%%%%%%%%%%%%%%%%%%%%%%%%%%%%%%%%%%%%%%%%%%%%%%%%%%%%%%%%%%%%%%%%%%%
%
%:::Change-Log:::
% Versionierung erfolgt auf folgende Gegebenheiten: -1. Release Versionen
%                                                   -2. Neue Kapitel
%                                                   -3. Fehlerkorrekturen
%
% 0.0.0       Erstellung der Datei
%%%%%%%%%%%%%%%%%%%%%%%%%%%%%%%%%%%%%%%%%%%%%%%%%%%%%%%%%%%%%%%%%%%%%%%%%%%%%%%
\chapter{Unit-Testing}\label{ch:unittesting}
	Ein Unit-Test ist ein St"uck Code der einen bestimmten Teil einer Software testet. Dieser Test-Code kann, ist er einmal geschrieben, beliebig oft wiederholt werden. Somit kann bei "Anderungen in der zu testenden Software der entsprechende Unit-Test durchgef"uhrt werden, um sicher zu stellen, dass keine neuen Fehler vorhanden sind. 
	%
	\section{Warum Unit-Tests}\label{s:warumunittests}
		Heutzutage ist es in der Industrie praktisch unabdingbar, Unit-Test f�r die entwickelte Software zu erstellen. Meist werden diese Test bereits in der Entwicklungsphase parallel oder gar vor der eigentlichen Software definiert und umgesetzt. Besonders wichtig sind Unit-Test in Software an der ganze Teams arbeiten. Hat ein Softwareentwickler Code geschrieben, so kann dieser "uber Nacht getestet werden. Bei erfolgreichem Test kann der Code am n�chsten morgen f�r die "ubrigen Teammitglieder zur Verf"ugung gestellt werden.\par
		%
		Unit-Test erh�hen dadurch die Qualit"at des Codes und decken neu implementierte Fehler fr�hzeitig und k"onnen somit schnell und kosteng"unstig behoben werden. Dadurch wird auch das Vertrauen in die Software erh�ht.   
		%
	\section{Einsatzgebiete}\label{s:einsatzgebiete}
		Der Begriff Unit-Test bezieht sich nicht zwingend auf Software-Tests. Es kann auch bedeuten, dass zum Beispiel Hardwarekomponenten getestet werden. Dieses Dokument bezieht sich jedoch ausschliesslich auf Unit-Test f�r Software, insbesondere auf Java-Code.\par
		%
		Mehr Informationen "uber Tests in diversen Softwaresprachen sind in Kapitel \ref{s:xunit} auf Seite \pageref{s:xunit} zu finden.