%%%%%%%%%%%%%%%%%%%%%%%%%%%%%%%%%%%%%%%%%%%%%%%%%%%%%%%%%%%%%%%%%%%%%%%%%%%%%%%
% Titel:   Einleitung
% Autor:   
% Datum:   
% Version: 0.0.0
%%%%%%%%%%%%%%%%%%%%%%%%%%%%%%%%%%%%%%%%%%%%%%%%%%%%%%%%%%%%%%%%%%%%%%%%%%%%%%%
%
%:::Change-Log:::
% Versionierung erfolgt auf folgende Gegebenheiten: -1. Release Versionen
%                                                   -2. Neue Kapitel
%                                                   -3. Fehlerkorrekturen
%
% 0.0.0       Erstellung der Datei
%%%%%%%%%%%%%%%%%%%%%%%%%%%%%%%%%%%%%%%%%%%%%%%%%%%%%%%%%%%%%%%%%%%%%%%%%%%%%%%
\chapter{Einleitung}\label{ch:einleitung}
	In vielen Bereichen der Softwareentwicklung ist es heutzutage unumg�nglich sogenannte Unit-Tests zu entwickeln, um die Codequalit�t zu gew�hrleisten. Mit Hilfe von Unit-Tests kann der Code zu jeder Zeit automatisch getestet werden, so dass Fehler bei "Anderungen am Code rechtzeitig und somit kosteng�nstig behoben werden k�nnen.\par
	%
	Im Internet ist eine Vielzahl an Testumgebungen zu finden, die das erstellen von Unit-Tests erleichtern. Eine beliebte und viel genutzte Test-Framework-Familie ist \textsf{xUnit}. In diesem Dokument wird vorwiegend auf das Framework JUnit, welches in der Java-Programmierung eingesetzt wird, genauer eingegangen. Das Test-Framework \textsf{JUnit} bietet dem Benutzer allerhand Funktionen, um m�glichst effizient Unit-Tests umzusetzen.\par
	%
	Da Unit-Testing leider kein Platz im Lehrplan der \gls{ac:bfh} bekommen hat, soll diese Einleitung den Studenten einen m�glichst einfachen Einstieg in diese Thematik bieten. Nach der Einf�hrung in das Unit-Testing werden in folgenden Kapiteln die wichtigsten Schritte und M"oglichkeiten von \textsf{JUnit} kurz erkl�rt und mit einigen Anwendungsbeispiele veranschaulicht. Anschliessend wird auf die Erweiterung \textsf{EclEmma} eingegangen, welche eine gute Erg"anzung zum JUnit-Framework darstellt.
