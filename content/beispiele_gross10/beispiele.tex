%%%%%%%%%%%%%%%%%%%%%%%%%%%%%%%%%%%%%%%%%%%%%%%%%%%%%%%%%%%%%%%%%%%%%%%%%%%%%%%
% Titel:   Beispiele
% Autor:   Simon grossenbacher
% Datum:   22.09.2013
% Version: 1.0.1
%%%%%%%%%%%%%%%%%%%%%%%%%%%%%%%%%%%%%%%%%%%%%%%%%%%%%%%%%%%%%%%%%%%%%%%%%%%%%%%
%
%:::Change-Log:::
% Versionierung erfolgt auf folgende Gegebenheiten: -1. Release Versionen
%                                                   -2. Neue Kapitel
%                                                   -3. Fehlerkorrekturen
%
% 1.0.0       Erstellung der Datei
% 1.0.1       Bilder im Abschnitt "Verlinkungen" hinzugef�gt
%%%%%%%%%%%%%%%%%%%%%%%%%%%%%%%%%%%%%%%%%%%%%%%%%%%%%%%%%%%%%%%%%%%%%%%%%%%%%%%    
\chapter{Beispiele}\label{ch:beispiele}
    Das ist ein Beispiel-Kapitel.
    %
    %
    %Ueberschrift 1
    \section{�berschrift 1}\label{s:uebschrift_1}
        Das ist eine grosse �berschrift!\par
        N�chster Absatz.
        %
        %Ueberschrift 2
        \subsection{�berschrift 2}\label{ss:ueberschrift_2}
            Dies ist eine kleinere �berschrift!
            %
            \subsubsection{�berschrift 3}\label{sss:uberschrift_3}
                Noch eine kleinere �berschrift!
                %
                \paragraph{Paragraph} F�r kleine Zwischentitel eignet sich der Paragraph.
    %
    %
    %Aufz�hlungen
    \section{Aufz�hlungen}\label{s:aufzaehlungen}
        Es gibt mehrere M�glichkeiten:
        \begin{itemize}
            \item Hallo
            \item Tsch�ss
            %
            \begin{itemize} %[parsep=1pt, topsep=-10pt] Fuer vertikale Abstaende zwischen den Elementen
	            \item Unteraufz�hlung
            \end{itemize}
        \end{itemize}
        \begin{enumerate}
            \item Eis
            \item Zw�i
        \end{enumerate}
        \begin{description}
            \item[Ich] Du Ich Du 
            \item[Er] Sie Er Sie
        \end{description}
    %
    %
    %Formeln
    \section{Formeln}\label{s:formeln}
        Hier bietet sich die Funktion \texttt{formula} an.
        \formula{
            U=R\cdot I
        }{
            U & Spannung in Volt\\
            R & Widerstand\\
            I & Strom}[eq:uri]
        Es k�nnen auch mehrere Formeln zusammengefasst werden (man beachte das \textbf{\&} f�r die Ausrichtung)!
        \formula{
            U&=R\cdot I\\
            R&= \frac{U}{I}
        }{
            U & Spannung in Volt\\
            R & Widerstand\\
            I & Strom}[eq:uri2]
    %
    %
    %
    \section{Theoreme}\label{s:theoreme}
    	Manchmal ist es n�tzlich Theoremen oder Definition visuell hervorzuheben. Dazu muss das Paket \texttt{amsthm} eingebunden und in der Pr�ambel ein Theorem mit \texttt{newtheorem\{name\}\{Printed output\}} definiert werden. Das Ergebnis ist unterhalb ersichtlich in Definition \vref{thm:dokument}.
    	%
    	\begin{definition}\label{thm:dokument}
	    	Das ist meine Definition von einer sauberen Dokumentvorlage
    	\end{definition}
    %        
    %
    %Bilder
    \section{Bilder}\label{s:bilder}
        Bilder k�nnen mit der Funktion \texttt{image} ein gef�gt werden (siehe Abbildung \vref{abb:bfh}). Sollen zwei oder mehr Bilder nebeneinander angeordnet werden, so empfiehlt sich die \texttt{subfigure} Umgebung einzusetzen oder die Funktion \texttt{imageTwo} zu verwenden.
        \image{content/beispiele/image/bfh_logo}{scale=1}{htbp}[Unterschrift f�r unter dem Bild \cite{img:bfh}][Bildbeschreibung f�r im Abbildungsverzeichnis][abb:bfh]
        \imageTwo{content/beispiele/image/bfh_logo}{content/beispiele/image/bfh_logo}{scale=1}{htbp}[Unterschrift f�r unter dem Bild \cite{img:bfh}][Bildbeschreibung f�r im Abbildungsverzeichnis][abb:bfh]
    %
    %
    %Tabellen
    \section{Tabellen}\label{s:tabellen}
        Tabellen in \LaTeX sind ziemlich umst�ndlich. Daher empfiehlt sich ein entsprechendes Plugin in der Office-Umgebung, die ein \LaTeX -Export erm�glichen. Weiter sollten der �bersicht zu liebe alle Tabellen in separaten Dateien verwaltet werden. Beispiele sind \vrefrange{tab:tabelle}{tab:tabelleTwo} ersichtlich.
        %
        \begin{table}[htbp]
     \centering
     \caption{Tabelle 1}
     \label{tab:tabelle}
     \begin{tabular}{|l|l|l|} 
         \hline
         \rowcolor{bfhblue}
         \textcolor{white}{Spalte 1} & \textcolor{white}{Spalte 2} & \textcolor{white}{Spalte 3}\\
         \hline
         Ich & bin & da \\
         \hline
         \multicolumn{2}{|l|}{Zwei Spalten vereinen} & Das geht!\\
         \hline
     \end{tabular}  
\end{table}
        \begin{table}[htbp]
     \centering
	 \caption{Tabelle 2}
	 \label{tab:tabelleTwo}  
     \begin{tabular}{lrr} 
     \toprule
     \multicolumn{2}{c}{Studium}\\ \cmidrule[3pt]{1-2}
     Fach & Dauer & Einkommen (\euro{})\\ 
     \midrule 
     Info & 2 & 12,75 \\ \addlinespace
     MST & 6 & 8,20 \\ \addlinespace
     VWL & 14 & 10,00\\ 
     \bottomrule
     \end{tabular}
\end{table}
        %
        \begin{itemize}
            \item Falls Bilder in Tabellen n�tig sein sollten, Befehl \texttt{imagetotab} verwenden      
            \item Tabellen �ber mehrere Seiten sind mit der \texttt{longtable} m�glich
            \item Ist eine Tabelle breiter als es der Satzspiegel erlaubt, so kann sie mit \texttt{sidewaystable}\footnote{anstatt die \texttt{table}-Umgebung} des Pakets \texttt{rotating} gedreht werden
        \end{itemize}
    %
    %
    %Code
    \section{Code}\label{s:code}
        \subsection{ANSI C}\label{ss:c}
            \begin{lstlisting}[style=C,caption={Hallo Welt},label={list:hallowelt}]
printf("Hallo Welt"); //Dummy Funktion
            \end{lstlisting}  
        %
        % 
        \subsection{MATLAB}\label{ss:matlab}
            \begin{lstlisting}[style=Matlab,caption={Sinnlos},label={list:sinnlos}]
close all
clear all
t = [0:1000];
plot(t,t); %Dummy-Plot
            \end{lstlisting}  
    %
    %
    %Verlinkungen
    \section{Verlinkungen}\label{s:verlinkungen}
        \begin{description}
            \item[Fussnoten] Fussnoten werden mit dem Kommando \texttt{footnote} erstellt\footnote{Ich bin eine normale Fussnote}
            %
            \item[Verweise] Es kann auf alle Labels verwiesen werden. Dazu dienen die Befehle \texttt{vref}, \texttt{vrefrange\{\}\{\}}, \texttt{vpageref} und \texttt{vpagerefrange\{\}\{\}}: Der Abschnitt \vref{s:verlinkungen} befindet sich \vpageref{s:verlinkungen}. Das \texttt{varioref} Paket unterscheidet dabei ob sich ein Label auf der aktuellen, vorherigen oder n�chsten Seite befindet und passt seine Ausgabe dementsprechend an. Soll anstatt auf die Nummer eines Labels auf seinen Text verwiesen werden, so kann \texttt{nameref} eingesetzt werden: \nameref{s:verlinkungen}.\par 
            %
            Allgemein gilt f�r die einzelnen Dokumentkomponente folgende Bezeichnungen:
            \begin{description}
	            \item[chapter] Kapitel
	            \item[section - subsubsection] Abschnitt
	            \item[paragraph] Paragraph
	            \item[image] Abbildung
	            \item[tabularX] Tabelle
	            \item[formula] Gleichung
            \end{description}
            %
            \item[Quellen] Um auf Quellen zu verweisen, muss das Label bekannt sein (wird in der Datei \texttt{bibliography.bib} festgelegt). Anschliessend muss im \LaTeX~Editor \texttt{BibTex} ausgef�hrt werden. Daraufhin kann der Befehl \texttt{cite} eingesetzt werden \cite{lit:bsp} oder \cite[S.11]{lit:online}. Evtl. muss das Dokument mehrmals compiliert werden. F�r das bessere Verwalten der Quellen, kann bei Bedarf auch ein entsprechender Editor verwendet werden\footnote{z.B. JabRef \url{http://jabref.sourceforge.net/}}.
            \image{content/beispiele/image/bibtex}{scale=.5}{htbp}[Literaturverzeichnis erstellen]
            \image{content/beispiele/image/bib}{scale=.5}{htbp}[Ausschnitt aus der \texttt{bibliography.bib} Datei]
            %
            \item[Stichwortverzeichnis] W�rter f�r das Verzeichnis m�ssen mit \texttt{index} bekannt gemacht werden \index{Index}. F�r die Zusammenstellung des Verzeichnisses muss im \LaTeX~Editor \texttt{MakeIndex} durchgef�hrt werden.
            \image{content/beispiele/image/index}{scale=.5}{htbp}[Index erstellen][abb:bsp]
            %
            \item[Glossar] Definieren in den Dateien \texttt{glossary.text} und \texttt{acronyms.tex} und anschliessend mit \texttt{gls} verwenden\footnote{Achtung: \texttt{makeindex -s \%.ist -o \%.gls \%.acn \%.glo \%.idx}}. \gls{ac:bfh} \gls{g:mammut}
        \end{description}
    %
    %
    %Diverses
    \section{Diverses}\label{s:diverses}
        %
        %TODOs
        \subsection{TODOs}\label{ss:todos}
            Mit dem Befehl \texttt{todo} k�nnen noch unfertige Teile \todo{Noch nicht fertig} markiert werden.
        %
        %Randnotizen
        \subsection{Randnotizen}\label{ss:randnotizen}
        	Der Befehl \texttt{mpar} erm�glicht das Anbringen von Randnotizen. \mpar{Das ist eine Randnotiz}
       	%
       	%Einheiten
       	\subsection{Einheiten}
       		In einem tech. Bericht m�ssen immer wieder SI-Einheiten verwendet werden. Dazu gilt es das Paket \texttt{unit} zu nutzen.
       		%
       		\begin{itemize}
	       		\item Alle \unit[10]{cm}
	       		\item Er f�hrt \unitfrac[20]{m}{s} schnell
       		\end{itemize}	
