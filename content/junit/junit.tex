%%%%%%%%%%%%%%%%%%%%%%%%%%%%%%%%%%%%%%%%%%%%%%%%%%%%%%%%%%%%%%%%%%%%%%%%%%%%%%%
% Titel:   JUnit
% Autor:   
% Datum:   
% Version: 0.0.0
%%%%%%%%%%%%%%%%%%%%%%%%%%%%%%%%%%%%%%%%%%%%%%%%%%%%%%%%%%%%%%%%%%%%%%%%%%%%%%%
%
%:::Change-Log:::
% Versionierung erfolgt auf folgende Gegebenheiten: -1. Release Versionen
%                                                   -2. Neue Kapitel
%                                                   -3. Fehlerkorrekturen
%
% 0.0.0       Erstellung der Datei
%%%%%%%%%%%%%%%%%%%%%%%%%%%%%%%%%%%%%%%%%%%%%%%%%%%%%%%%%%%%%%%%%%%%%%%%%%%%%%%
\chapter{JUnit}\label{ch:junit}
	%
	\section{xUnit}\label{s:xunit}
		\textsf{JUnit} geh�rt zu der Framework-Familie \textsf{xUnit}. Verschiedenen Unit-Test Umgebungen f�r verschiedenen Programmiersprachen werden mit dem Begriff \textsf{xUnit} zusammengefasst. Das x in \textsf{xUnit} stellt dabei ein Platzhalter zur Identifikation der zu testenden Programmiersprache des Frameworks dar. In den Meisten F"allen wird das x mit den ersten Buchstaben der Programmiersprache ersetzt (z.B. CUnit f�r C oder in unserem Fall JUnit f�r Java).\par
		%
		\textsf{xUnit} umfasst bei weitem nicht alle Frameworks f�r das Unit-Testing. Im Internet ist eine grosse Auswahl an verf�gbaren Alternativen zur \textsf{xUnit} Familie zu finden. Unter \hyperlink{http://de.wikipedia.org/wiki/Liste_von_Modultest-Software}{Wikipedia - Liste von Modultest-Software}\cite{lit:unittestlist} ist eine Auflistung von vorhandenen Test-Frameworks f�r diverse Programmiersprachen vorhanden.   
		%
	\section{Funktionsweise}\label{s:funktionsweise}
		%
	\section{Test Case}\label{s:testcase}
		%
	\section{Annotationen}\label{s:annotationen}
